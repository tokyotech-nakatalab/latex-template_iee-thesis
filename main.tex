\documentclass[bachelor,supervisor]{keieithesis} %卒論の場合"master"を"bachelor"に,博論の場合"doctor"に変更
\usepackage{fancybox,calc}
\usepackage{amsthm,amsmath,txfonts,bm,latexsym, colortbl,bm}
\usepackage{ascmac,amssymb,amsfonts,booktabs,multirow}
\usepackage{algorithm, algorithmic} %アルゴリズム
\usepackage{url} %URLの表記用
\usepackage[dvipdfmx]{graphicx}
\allowdisplaybreaks[1]



\newcommand{\argmax}{\mathop{\rm arg~max}\limits}
\newcommand{\argmin}{\mathop{\rm arg~min}\limits}
\newtheorem{theorem}{定理}[chapter]
\newtheorem{proposition}{命題}[chapter]
\newtheorem{remark}{注意}[chapter]
\newtheorem{definition}{定義}[chapter]
\newtheorem{example}{例}[chapter]
\newtheorem{corollary}{系}[chapter]
\renewcommand{\proofname}{\bf 証明}
\newtheorem{lemma}{補題}[chapter]


\title{タイトル}%タイトル
%\titlewidth{}% タイトル幅 (指定するときは単位つきで)不要ならコメントアウト
\etitle{title}%英文タイトル(必要はない)
\author{西之園 萌絵}%氏名
\eauthor{Name}%英文氏名
\studentid{21B00000}%学籍番号
\supervisor{犀川 創平 教授}% 指導教官名(役職まで含めて書く)
\handin{2023}{2}% 提出年, 月

\begin{document}
\maketitle
\frontmatter

% 英文概要(修士のみ)
\begin{eabstract}
abstract
\end{eabstract}

% 和文概要
\begin{abstract}
概要

\end{abstract}

\tableofcontents

\mainmatter
%本文領域
% \include{file}などで章ごとにファイルを分割してもよし
\setcounter{chapter}{0}
\chapter{序論}
序章で書くこと(少なくとも、研究の背景と目的、既存研究との関係あるいは研究の位置づけ、研究の具体的な内容、得られた結果など)をよく考え、それぞれをひとつ以上の段落(パラグラフ)を使うようにして、段落の構成を決める。
各段落で何を述べるか決め、いくつかの文で構成します。各文を正しい日本語で記述し(特に主語と述語の関係)、文のつながりを意識して書くこと。
あまり長い文は使わない。
また、長い文を使うときなどには、文の意味が不明とならないように注意すること。

参考文献の引用には \verb|\cite{}| コマンドを用いる。
Nakata \textit{et al.} \cite{Nakata2003} といった具合で記載する。
複数の文献を一度に参照したい場合は \cite{Nakata2003,Fukuda2001} とする。


序章で書くこと(少なくとも、研究の背景と目的、既存研究との関係あるいは研究の位置づけ、研究の具体的な内容、得られた結果など)をよく考え、それぞれをひとつ以上の段落(パラグラフ)を使うようにして、段落の構成を決め, 各段落で何を述べるか決め、いくつかの文で構成します。各文を正しい日本語で記述し(特に主語と述語の関係)、文のつながりを意識して書くこと。

\setcounter{chapter}{1}
\chapter{章のタイトル}
各章の内容をよく考え、節の構成とそのタイトルを決める。各章のはじめに、その章の内容のサマリーを書く。
\section{節のタイトル}
各節の内容をよく考え、必要ならば小節の構成とそのタイトルを決める。長い節は、節のはじめに、その節の内容のサマリーを書く。各節あるいは小節の段落の構成を決める。(以下序章の書き方を参考)
\setcounter{chapter}{2}
\chapter{章のタイトル}
各章の内容をよく考え、節の構成とそのタイトルを決める。各章のはじめに、その章の内容のサマリーを書く。
\section{節のタイトル}
\setcounter{chapter}{3}
\chapter{章のタイトル}
各章の内容をよく考え、節の構成とそのタイトルを決める。各章のはじめに、その章の内容のサマリーを書く。
\section{節のタイトル}
各節の内容をよく考え、必要ならば小節の構成とそのタイトルを決める。長い節は、節のはじめに、その節の内容のサマリーを書く。各節あるいは小節の段落の構成を決める。(以下序章の書き方を参考)
\setcounter{chapter}{4}
\chapter{章のタイトル}
各章の内容をよく考え、節の構成とそのタイトルを決める。各章のはじめに、その章の内容のサマリーを書く。
\section{節のタイトル}
各節の内容をよく考え、必要ならば小節の構成とそのタイトルを決める。長い節は、節のはじめに、その節の内容のサマリーを書く。各節あるいは小節の段落の構成を決める。(以下序章の書き方を参考)
\setcounter{chapter}{5}
\chapter{章のタイトル}
各章の内容をよく考え、節の構成とそのタイトルを決める。各章のはじめに、その章の内容のサマリーを書く。
\section{節のタイトル}
各節の内容をよく考え、必要ならば小節の構成とそのタイトルを決める。長い節は、節のはじめに、その節の内容のサマリーを書く。各節あるいは小節の段落の構成を決める。(以下序章の書き方を参考)

\backmatter% ここから後付,謝辞やAppendix, 参考文献など
\chapter*{謝辞}

謝辞
\begin{flushright}
2023年~2月~~~名前~名前
\end{flushright}



% 以下参考文献
% bibファイルを利用する場合は以下の形式
% \include{bibliography}
\bibliography{bibliography} %hoge.bibから拡張子を外した名前
\bibliographystyle{junsrt} %junsrtは変更不要

\appendix
%Appendix領域
% \chapter{}とすると,付録Aが生成される.名前は必要なし.2つ以上だと付録B, Cと続く.本文と同様\include{file}の利用も可能
\chapter{付録の章のタイトルはここに}

\end{document}


\setcounter{chapter}{0}
\chapter{序論}
序章で書くこと(少なくとも、研究の背景と目的、既存研究との関係あるいは研究の位置づけ、研究の具体的な内容、得られた結果など)をよく考え、それぞれをひとつ以上の段落(パラグラフ)を使うようにして、段落の構成を決める。
各段落で何を述べるか決め、いくつかの文で構成します。各文を正しい日本語で記述し(特に主語と述語の関係)、文のつながりを意識して書くこと。
あまり長い文は使わない。
また、長い文を使うときなどには、文の意味が不明とならないように注意すること。

参考文献の引用には \verb|\cite{}| コマンドを用いる。
Nakata \textit{et al.} \cite{Nakata2003} といった具合で記載する。
複数の文献を一度に参照したい場合は \cite{Nakata2003,Fukuda2001} とする。


序章で書くこと(少なくとも、研究の背景と目的、既存研究との関係あるいは研究の位置づけ、研究の具体的な内容、得られた結果など)をよく考え、それぞれをひとつ以上の段落(パラグラフ)を使うようにして、段落の構成を決め, 各段落で何を述べるか決め、いくつかの文で構成します。各文を正しい日本語で記述し(特に主語と述語の関係)、文のつながりを意識して書くこと。
